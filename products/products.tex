\documentclass[12pt, letterpaper]{article}

% text stuhh
\newcommand{\documentname}{\textsl{Note}}

% math stuhh
\newcommand{\given}{\,|\,}
\newcommand{\dd}{\mathrm{d}}
\newcommand{\T}{^{\!\mathsf{T}}}
\newcommand{\inv}{^{-1}}

% typesetting stuff
\addtolength{\topmargin}{-0.75in}
\addtolength{\textheight}{1.5in}
\raggedbottom\sloppy\sloppypar\frenchspacing

\begin{document}

\section*{Products of dependent multivariate Gaussians}

\noindent
\textbf{David W. Hogg}, \textbf{Adrian Price-Whelan}

\paragraph{Abstract:} Foo and bar.

\section{Marginalization by refactorization}

Imagine that we are doing an inference using data $y$ (which is a
$N$-dimensional vector, say).
We are trying to learn parameters $\theta$ and $P$ (the former a
$K$-dimensional vector and the latter an arbitrary vector, list,
or blob.
Whether we are Bayesian or frequentist, the inference is based on
a likelihood function, or probability for the data given parameters
\begin{equation}
\mbox{likelihood} \quad p(y\given\theta,P)
\quad .
\end{equation}

Now let's imagine that the parameters $\theta$ are either nuisance
parameters, or else easily marginalized, so we want to marginalize
them out.
This will leave us with a lower-dimensional marginalized likelihood
function
\begin{equation}
\mbox{marginalized likelihood} \quad p(y\given P)
\quad .
\end{equation}
That's good, but the marginalization comes at a cost:
We have to choose a prior
\begin{equation}
\mbox{prior on nuisances} \quad p(\theta\given P)
\end{equation}
on the nuisances.
This is the basis for the claim (stated elsewhere; HOGG) that
inference requires a likelihood function, and priors on the nuisances.
It does not require a prior on everything, contrary to some statements
in the pedagogical literature (for example, HOGG).
We have said ``$p(\theta\given P)$'' because this prior may depend on
the parameters $P$. But it certainly doesn't have to.

To perform the marginalization, we have two choices.
We can either do an integral:
\begin{equation}
p(y\given P) = \int p(y\given\theta, P)\,p(\theta\given P)\,\dd\theta
\quad ,
\end{equation}
where the integral is implicitly over the entire domain of the
nuisances $\theta$ (or the entire support of the prior).
Or we can do a clever re-factorization of this form:
\begin{equation}
p(y\given\theta,P)\,p(\theta\given P)
 = p(\theta\given y,P)\,p(y\given P)
\quad .
\end{equation}
That is, in certain magical circumstances it is possible to do this
re-factorization without explicitly doing any integral.
When this is true, the marginalization is sometimes far easier than
the relevant integral.

One such magical circumstance can arise when the two probability
distributions---the likelihood and the prior---are both Gaussian in
form, and when the model is linear.
In detail we will assume
\begin{enumerate}
\item
the likelihood $p(y\given\theta, P)$ is a Gaussian in $y$,
\item
the prior $p(\theta\given P)$ is a Gaussian in $\theta$,
\item
the mean of the likelihood Gaussian depends linearly on the nuisance
parameters $\theta$, and
\item
the nuisance parameters $\theta$ don't enter the likelihood anywhere
other than in the mean.
\end{enumerate}
In equations, this becomes:
\begin{equation}
p(y\given\theta, P) = N(y\given M\cdot\theta, C)
\end{equation}
\begin{equation}
p(\theta\given P) = N(\theta\given \mu, V)
\quad ,
\end{equation}
where
$N(x\given m,V)$ is the multivariate Gaussian pdf for a vector $x$
given a mean vector $m$ and a variance tensor $V$,
$M$ is a $N\times K$ rectangular design matrix (which depends, in
general, on the non-nuisance parameters $P$),
$C$ is a $N\times N$ covariance matrix of uncertainties for the
data (diagonal if the data dimensions are independent).
That is, the likelihood is a Gaussian with a mean that depends
linearly on the nuisance parameters $\theta$, and
$\mu$ and $V$ are the $K$-vector mean and $K\times K$ variance tensor
for the Gaussian prior.

In this incredibly restrictive---but also surprisingly
common---situation, the re-factored pdfs $p(\theta\given y, P)$ (also
known as the posterior) and $p(y\given P)$ (the marginalized
likelihood) will also both be Gaussian.
Obtaining the specific forms of these Gaussians is the object of this
\documentname.

\section{Products of Gaussians}

On the internets, there are many documents, slide decks, and videos
that explain products of Gaussians in terms of other Gaussians.
The vast majority of these consider either the univariate case (where
the data $y$ and the parameter $\theta$ are both simple scalars, which
is completely uninteresting), or the same-dimension case (where the data
$y$ and the parameter vector $\theta$ are the same length, which never
occurs in practice).
Here we solve this problem in the general case:
The inputs are multivariate (vectors) and the two Gaussians we are
multiplying live in spaces of different dimensions.
That is, we solve the following problem:

\paragraph{Problem:}
Find $N$-vector $a$, $N\times N$ variance tensor $A$, $K$-vector $b$,
and $K\times K$ variance tensor $B$ such that
\begin{equation}
N(y\given M\cdot\theta, C)\,N(\theta\given\mu, V) = N(y\given a,
A)\,N(\theta\given b, B) \quad ,
\end{equation}
and such that $a$ and $A$ don't depend on $\theta$ at all.

\paragraph{Solution:}
\begin{equation}
A = C + M \cdot V \cdot M\T
\end{equation}
\begin{equation}
a = M \cdot \mu
\end{equation}
\begin{equation}
B\inv = V\inv + M\T \cdot C\inv \cdot M
\end{equation}
\begin{equation}
b = B \cdot (V\inv \cdot \mu + M\T \cdot C\inv \cdot y)
\quad .
\end{equation}

\paragraph{Solution notes:}
\begin{itemize}
\item
We found this by completing the square. Actually we didn't really; we just
used clever arguments.
\item
Check out both the units and the dimensions of $M$ and where it appears!
\item
Note that $a$ is the prior-optimal value for the data $y$.
\item
Note that $b$ is the MAP value for the nuisance parameters $\theta$.
\item
Check out how we are zero-safe in the variance tensors.
\item
Comment about determinants of the matrices?
\end{itemize}

\end{document}
